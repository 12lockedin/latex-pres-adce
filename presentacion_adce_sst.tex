\documentclass[aspectratio=169, 10pt]{beamer} 
% aspectratio=169 fuerza el formato panorámico

% --- TEMA Y FUENTES ---
\usetheme{metropolis}
% Opciones del tema: Bloques con fondo de color y barra de progreso
\metroset{block=fill, progressbar=frametitle, background=light}

\usepackage[sfdefault]{raleway} % Fuente moderna sans-serif
\usepackage[T1]{fontenc}
\usepackage[utf8]{inputenc}
\usepackage[spanish]{babel}

% --- PAQUETES GRÁFICOS ---
\usepackage{graphicx}
\usepackage{booktabs}
\usepackage{tikz}
\usetikzlibrary{positioning, arrows.meta, shapes, calc} % Librerías para mejores diagramas

% --- COLORES PERSONALIZADOS ---
\definecolor{primary}{RGB}{0,102,204}      % Azul corporativo
\definecolor{secondary}{RGB}{51,153,255}    % Azul claro
\definecolor{accent}{RGB}{255,87,34}        % Naranja acento
\definecolor{darkgray}{RGB}{60,60,60}

% Asignación de colores al tema
\setbeamercolor{palette primary}{bg=primary, fg=white}
\setbeamercolor{frametitle}{bg=primary, fg=white}
\setbeamercolor{alerted text}{fg=accent}
\setbeamercolor{progress bar}{fg=accent, bg=primary!30}

% --- INFORMACIÓN DE LA PRESENTACIÓN ---
\title{Transformadores de Estado Sólido (SST)}
\subtitle{Innovación tecnológica en control de tensión}
\author{Tu Nombre}
\institute{ADCE}
\date{\today}

\begin{document}

% 1. PORTADA
\begin{frame}
    \titlepage
\end{frame}

% 2. ÍNDICE
\begin{frame}{Contenido}
    \setbeamertemplate{section in toc}[sections numbered]
    \tableofcontents
\end{frame}

% --- SECCIÓN 1 ---
\section{Introducción}

\begin{frame}{¿Qué vamos a ver hoy?}
    \begin{columns}[T] % Alineación superior
        \column{0.5\textwidth}
        \vspace{1em}
        \begin{itemize}
            \item Una tendencia tecnológica emergente en el campo del control de tensión.
            \item Los \textbf{Transformadores de Estado Sólido (SST)}.
            \item Conocidos como \textit{Solid State Transformers} en inglés.
        \end{itemize}
        
        \column{0.5\textwidth}
        \begin{figure}
            \centering
            % NOTA: Reemplaza 'example-image-a' con tu archivo 'sst_conceptual.png'
            \includegraphics[width=0.9\textwidth]{example-image-a}
            \caption{\footnotesize Ilustración conceptual de un SST moderno.}
        \end{figure}
    \end{columns}
\end{frame}

% --- SECCIÓN 2 ---
\section{Conceptos Previos}

\begin{frame}{Recordatorio 1: Transformador Convencional}
    \begin{columns}
        \column{0.5\textwidth}
        \textbf{¿Qué hace?}
        \begin{itemize}
            \item Transforma niveles de tensión en CA.
            \item Utiliza campos electromagnéticos.
            \item Dos devanados acoplados magnéticamente.
        \end{itemize}
        
        \vspace{0.5em}
        \textbf{Importancia:}
        \begin{itemize}
            \item Razón principal del uso de corriente alterna.
            \item Eficiencia energética en transmisión.
        \end{itemize}

        \column{0.5\textwidth}
        \begin{figure}
            \centering
            % Reemplaza con 'transformador_convencional'
            \includegraphics[width=0.8\textwidth]{example-image-b}
            \caption{\footnotesize Esquema de un transformador convencional.}
        \end{figure}
    \end{columns}
\end{frame}

\begin{frame}{Recordatorio 2: STATCOM}
    \begin{columns}
        \column{0.55\textwidth}
        \textbf{Función principal:}
        \begin{itemize}
            \item Control de tensión en la red.
            \item Mejora calidad del suministro.
            \item Intercambio de potencia reactiva.
        \end{itemize}
        
        \vspace{0.5em}
        \textbf{Características:}
        \begin{itemize}
            \item Conexión en \textbf{\textcolor{accent}{paralelo}}.
            \item Basado en VSC (\textit{Voltage Source Converters}).
            \item Consume solo potencia activa para pérdidas.
        \end{itemize}

        \column{0.45\textwidth}
        \begin{figure}
            \centering
            % Reemplaza con 'statcom_esquema'
            \includegraphics[width=0.9\textwidth]{example-image-c}
            \caption{\footnotesize STATCOM en paralelo a la red.}
        \end{figure}
    \end{columns}
\end{frame}

% --- SECCIÓN 3 ---
\section{Transformadores de Estado Sólido}

\begin{frame}{¿Qué son los SST?}
    \begin{block}{Definición}
        Los SST son equipos de paso que se conectan \textbf{\textcolor{accent}{en serie}} al flujo de potencia, similar a los transformadores convencionales, pero usando electrónica de potencia.
    \end{block}

    \vspace{1.5em}

    \begin{center}
    \begin{tikzpicture}[auto, node distance=2cm, >=Stealth]
        % Dibujo esquemático mejorado
        \node (redMT) [text width=1.5cm, align=center] {Red MT};
        \node (sst) [right=1cm of redMT, fill=secondary, text=white, minimum width=2.5cm, minimum height=1cm, rounded corners] {\textbf{SST}};
        \node (redBT) [right=1cm of sst, text width=1.5cm, align=center] {Red BT};
        
        \draw[very thick, primary, ->] (redMT) -- (sst);
        \draw[very thick, primary, ->] (sst) -- (redBT);
    \end{tikzpicture}
    \end{center}

    \vspace{1em}
    \centering
    \textbf{Diferencia clave:} STATCOM $\rightarrow$ Paralelo (Derivación) vs. SST $\rightarrow$ Serie (Paso).
\end{frame}

\begin{frame}{Funcionalidad Híbrida}
    \begin{center}
        \LARGE
        \textcolor{primary}{SST} = \textcolor{secondary}{Transformador} + \textcolor{accent}{Compensador}
    \end{center}

    \vspace{1.5em}

    \begin{columns}[t]
        \column{0.45\textwidth}
        \begin{exampleblock}{Como Transformador}
            \begin{itemize}
                \item Transforma tensión.
                \item Transfiere potencia activa.
                \item Aislamiento galvánico.
            \end{itemize}
        \end{exampleblock}

        \column{0.45\textwidth}
        \begin{alertblock}{Como Compensador}
            \begin{itemize}
                \item Control de factor de potencia.
                \item Regulación dinámica.
                \item Eliminación de armónicos.
            \end{itemize}
        \end{alertblock}
    \end{columns}
\end{frame}

% --- SECCIÓN 4 ---
\section{Características Principales}

\begin{frame}{Características Esenciales de los SST}
    \begin{enumerate}
        \item \textbf{Transformación de tensión} con transmisión de potencia activa.
        \vspace{0.3em}
        \item \textbf{Control de factor de potencia} activo.
        \vspace{0.3em}
        \item \textbf{Regulación dinámica de tensión:}
        \begin{itemize}
            \item Si entrada baja 10\% $\rightarrow$ salida mantiene 230V exactos.
            \item (Un trafo convencional no puede hacer esto sin tap-changers lentos).
        \end{itemize}
        \vspace{0.3em}
        \item \textbf{Reducción drástica de peso y volumen:}
        \begin{itemize}
            \item Opera a $\sim$20,000 Hz (vs 50 Hz convencional).
            \item Transformador físico mucho más pequeño.
        \end{itemize}
    \end{enumerate}
\end{frame}

\begin{frame}{Más Características Destacadas}
    \begin{columns}
        \column{0.6\textwidth}
        \begin{enumerate}
            \setcounter{enumi}{4}
            \item \textbf{Bus de corriente continua (DC):}
            \begin{itemize}
                \footnotesize
                \item Conexión directa de paneles solares.
                \item Carga directa de VE sin rectificador externo.
            \end{itemize}
            \item \textbf{Eliminación de armónicos.}
            \item \textbf{Clave para SmartGrids:}
            \begin{itemize}
                 \footnotesize
                \item Control total del flujo (bidireccional).
            \end{itemize}
        \end{enumerate}
        
        \column{0.4\textwidth}
        \begin{alertblock}{Aplicación}
            Principalmente en redes de \textbf{Media} y \textbf{Baja Tensión} (debido a límites actuales de los semiconductores).
        \end{alertblock}
    \end{columns}
\end{frame}

% --- SECCIÓN 5 ---
\section{Arquitectura del SST}

\begin{frame}{Esquema Típico de un SST}
    \begin{figure}
        \centering
        % Reemplaza con 'sst_bloques'
        \includegraphics[width=0.85\textwidth, height=4cm]{example-image}
        \caption{\footnotesize Tres etapas: Rectificador AC/DC + Convertidor DC/DC (con HFT) + Inversor DC/AC.}
    \end{figure}
\end{frame}

\begin{frame}{Etapa 2: El "Corazón" (DC/DC Alta Frecuencia)}
    \begin{columns}
        \column{0.55\textwidth}
        \begin{block}{Componentes}
        \begin{itemize}
            \item Inversor DC/AC (alta frecuencia).
            \item \textbf{Transformador de Alta Frecuencia (HFT)}.
            \item Rectificador AC/DC.
        \end{itemize}
        \end{block}
        
        \vspace{0.5em}
        \textbf{Ventaja del HFT:}
        \small
        Al operar a $\sim$20 kHz, el núcleo magnético necesario es minúsculo comparado con uno de 50 Hz, manteniendo el aislamiento galvánico.
        
        \column{0.4\textwidth}
        \begin{figure}
            \centering
            % Reemplaza con 'hft_comparacion'
            \includegraphics[width=0.8\textwidth]{example-image-b}
            \caption{\footnotesize HFT vs Trafo convencional.}
        \end{figure}
    \end{columns}
\end{frame}

\begin{frame}{Proceso de Conversión Completo}
    \centering
    % Diagrama TikZ mejorado y rediseñado
    \begin{tikzpicture}[
        node distance=0.8cm and 0.5cm,
        auto,
        block/.style={rectangle, draw=primary, thick, fill=white, text width=1.4cm, align=center, rounded corners, minimum height=1cm, font=\scriptsize},
        arrow/.style={-Stealth, thick, primary}
    ]

        % Nodos
        \node[block] (rect) {Rectific.\\AC/DC};
        \node[block, right=of rect] (invHF) {Inv. HF\\DC/AC};
        \node[block, right=of invHF, fill=secondary!20, draw=secondary] (trafo) {\textbf{HFT}\\20kHz};
        \node[block, right=of trafo] (rectHF) {Rect. HF\\AC/DC};
        \node[block, right=of rectHF] (inv) {Inversor\\DC/AC};

        % Flechas y etiquetas
        \draw[arrow] (rect) -- node[above, font=\tiny]{DC} (invHF);
        \draw[arrow] (invHF) -- node[above, font=\tiny]{AC HF} (trafo);
        \draw[arrow] (trafo) -- node[above, font=\tiny]{AC BT} (rectHF);
        \draw[arrow] (rectHF) -- node[above, font=\tiny]{DC} (inv);

        % Entradas y salidas
        \node[left=0.5cm of rect] (in) {\textbf{MT 50Hz}};
        \draw[arrow] (in) -- (rect);
        
        \node[right=0.5cm of inv] (out) {\textbf{BT 50Hz}};
        \draw[arrow] (inv) -- (out);

    \end{tikzpicture}

    \vspace{2em}
    \begin{alertblock}{La Clave}
        Pasar por \textbf{Alta Frecuencia} intermedia permite reducir el tamaño físico drásticamente.
    \end{alertblock}
\end{frame}

% --- SECCIÓN 6 ---
\section{Ventajas y Desafíos}

\begin{frame}{Resumen: Pros y Contras}
    \begin{columns}[t]
        \column{0.5\textwidth}
        \textbf{\textcolor{primary}{Ventajas}}
        \begin{itemize}
            \item[\checkmark] Tamaño y peso reducidos.
            \item[\checkmark] Control instantáneo de voltaje.
            \item[\checkmark] Bus DC integrado (Renovables/VE).
            \item[\checkmark] Bidireccionalidad.
        \end{itemize}

        \column{0.5\textwidth}
        \textbf{\textcolor{accent}{Desafíos}}
        \begin{itemize}
            \item[$\times$] Coste elevado (semiconductores).
            \item[$\times$] Pérdidas ligeramente mayores (etapas múltiples).
            \item[$\times$] \textbf{Pérdida de inercia:} Al no tener masa rotatoria, reduce la estabilidad natural de la red ante cambios bruscos de frecuencia.
        \end{itemize}
    \end{columns}
\end{frame}

% --- CIERRE ---
\section{Conclusiones}

\begin{frame}{Conclusiones}
    \begin{block}{Potencial Transformador}
        Los SST unifican la transformación de tensión y la calidad de onda en un solo equipo compacto e inteligente.
    \end{block}

    \vspace{1em}
    \textbf{Hoja de ruta:}
    \begin{itemize}
        \item Tecnología probada pero cara.
        \item Necesario reducir costes de carburo de silicio (SiC).
        \item Futuro estándar para las \textit{Smart Cities}.
    \end{itemize}
\end{frame}

\begin{frame}[standout]
    \Huge ¿Preguntas?
    
    \vspace{1em}
    \normalsize \url{email@instituto.com}
\end{frame}

\end{document}