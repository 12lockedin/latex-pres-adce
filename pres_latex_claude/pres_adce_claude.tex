\documentclass[aspectratio=169, 10pt]{beamer} 
% aspectratio=169 fuerza el formato panorámico

% --- TEMA Y FUENTES ---
\usetheme{metropolis}
% Opciones del tema: Bloques con fondo de color y barra de progreso
\metroset{block=fill, progressbar=frametitle, background=light}

\usepackage[sfdefault]{raleway} % Fuente moderna sans-serif
\usepackage[T1]{fontenc}
\usepackage[utf8]{inputenc}
\usepackage[spanish]{babel}

% --- PAQUETES GRÁFICOS ---
\usepackage{graphicx}
\usepackage{booktabs}
\usepackage{tikz}
\usetikzlibrary{positioning, arrows.meta, shapes, calc, shadows, decorations.pathmorphing, patterns, fit, backgrounds}
\usepackage{fontawesome5} % Para iconos modernos

% --- COLORES PERSONALIZADOS ---
\definecolor{primary}{RGB}{0,102,204}      % Azul corporativo
\definecolor{secondary}{RGB}{51,153,255}    % Azul claro
\definecolor{accent}{RGB}{255,87,34}        % Naranja acento
\definecolor{darkgray}{RGB}{60,60,60}
\definecolor{success}{RGB}{76,175,80}       % Verde éxito
\definecolor{warning}{RGB}{255,152,0}       % Amarillo advertencia

% Asignación de colores al tema
\setbeamercolor{palette primary}{bg=primary, fg=white}
\setbeamercolor{frametitle}{bg=primary, fg=white}
\setbeamercolor{alerted text}{fg=accent}
\setbeamercolor{progress bar}{fg=accent, bg=primary!30}

% --- INFORMACIÓN DE LA PRESENTACIÓN ---
\title{Transformadores de Estado Sólido (SST)}
\subtitle{Innovación tecnológica en control de tensión}
\author{Tu Nombre}
\institute{ADCE}
\date{\today}

\begin{document}

% 1. PORTADA MEJORADA
\begin{frame}
    \titlepage
    \begin{tikzpicture}[remember picture,overlay]
        % Elementos decorativos en la portada
        \fill[accent, opacity=0.1] (current page.south west) rectangle ([yshift=1cm]current page.south east);
        \node[anchor=south east, opacity=0.3] at ([xshift=-0.5cm, yshift=0.5cm]current page.south east) 
            {\includegraphics[width=3cm]{example-image-a}};
    \end{tikzpicture}
\end{frame}

% 2. ÍNDICE MEJORADO
\begin{frame}{Contenido}
    \setbeamertemplate{section in toc}[sections numbered]
    \tableofcontents
\end{frame}

% --- SECCIÓN 1 ---
\section{Introducción}

\begin{frame}{¿Qué vamos a ver hoy?}
    \begin{columns}[T]
        \column{0.5\textwidth}
        \vspace{1em}
        
        % Animación progresiva del contenido
        \begin{itemize}
            \item<1-> Una tendencia tecnológica \textbf{emergente} en el campo del control de tensión.
            \item<2-> Los \textbf{\textcolor{accent}{Transformadores de Estado Sólido (SST)}}.
            \item<3-> Conocidos como \textit{Solid State Transformers} en inglés.
        \end{itemize}
        
        \vspace{1em}
        \onslide<4->{
        \begin{alertblock}{Objetivo}
            Entender cómo esta tecnología revoluciona la distribución eléctrica.
        \end{alertblock}
        }
        
        \column{0.5\textwidth}
        \begin{tikzpicture}[scale=0.9, transform shape]
            % Diagrama conceptual mejorado de un SST
            \onslide<2->{
            % Caja principal del SST
            \draw[fill=secondary!20, draw=primary, line width=2pt, rounded corners] 
                (0,0) rectangle (4,3);
            \node[font=\Large\bfseries, primary] at (2,2.5) {SST};
            
            % Elementos internos
            \draw[fill=white, draw=primary] (0.5,0.3) rectangle (1.5,0.8);
            \node[font=\tiny] at (1,0.55) {AC/DC};
            
            \draw[fill=white, draw=primary] (1.8,0.3) rectangle (2.8,0.8);
            \node[font=\tiny] at (2.3,0.55) {DC/DC};
            
            \draw[fill=white, draw=primary] (3.1,0.3) rectangle (4.1,0.8);
            \node[font=\tiny] at (3.6,0.55) {DC/AC};
            
            % Conexiones
            \draw[-Stealth, thick] (-1,1.5) -- (0,1.5) node[midway, above, font=\scriptsize] {MT};
            \draw[-Stealth, thick] (4,1.5) -- (5,1.5) node[midway, above, font=\scriptsize] {BT};
            
            % Bus DC
            \draw[accent, line width=2pt] (0.5,1.3) -- (3.6,1.3);
            \node[accent, font=\tiny] at (2,1.6) {Bus DC};
            }
        \end{tikzpicture}
    \end{columns}
\end{frame}

% --- SECCIÓN 2 ---
\section{Conceptos Previos}

\begin{frame}{Recordatorio 1: Transformador Convencional}
    \begin{columns}
        \column{0.5\textwidth}
        \textbf{¿Qué hace?}
        \begin{itemize}
            \item<1-> Transforma niveles de tensión en CA.
            \item<2-> Utiliza campos electromagnéticos.
            \item<3-> Dos devanados acoplados magnéticamente.
        \end{itemize}
        
        \vspace{0.5em}
        \onslide<4->{
        \textbf{Importancia:}
        \begin{itemize}
            \item Razón principal del uso de corriente alterna.
            \item Eficiencia energética en transmisión.
        \end{itemize}
        }

        \column{0.5\textwidth}
        \begin{tikzpicture}[scale=0.8]
            \onslide<2->{
            % Núcleo magnético
            \draw[fill=darkgray!30, draw=darkgray, line width=1.5pt] 
                (-0.3,-2) rectangle (0.3,2);
            
            % Devanado primario
            \draw[primary, line width=2pt, decoration={coil, aspect=0.5, segment length=2mm, amplitude=3mm}, decorate] 
                (-0.3,-1.5) -- (-0.3,1.5);
            \node[primary, left] at (-0.3,2.2) {Primario};
            \draw[-Stealth, primary, thick] (-1.5,1.5) -- (-0.3,1.5);
            \draw[-Stealth, primary, thick] (-0.3,-1.5) -- (-1.5,-1.5);
            \node[primary, font=\small] at (-2.2,0) {$V_1$};
            
            % Devanado secundario
            \draw[secondary, line width=2pt, decoration={coil, aspect=0.5, segment length=2mm, amplitude=3mm}, decorate] 
                (0.3,-1) -- (0.3,1);
            \node[secondary, right] at (0.3,2.2) {Secundario};
            \draw[-Stealth, secondary, thick] (0.3,1) -- (1.5,1);
            \draw[-Stealth, secondary, thick] (1.5,-1) -- (0.3,-1);
            \node[secondary, font=\small] at (2.2,0) {$V_2$};
            
            % Flujo magnético
            \foreach \y in {-1.2,-0.4,0.4,1.2}
                \draw[-Stealth, accent, thick, opacity=0.5] (-0.15,\y) -- (0.15,\y);
            }
            
            \onslide<3->{
            \node[font=\footnotesize, text width=3cm, align=center] at (0,-2.8) 
                {$\frac{V_1}{V_2} = \frac{N_1}{N_2}$};
            }
        \end{tikzpicture}
    \end{columns}
\end{frame}

\begin{frame}{Recordatorio 2: STATCOM}
    \begin{columns}
        \column{0.55\textwidth}
        \textbf{Función principal:}
        \begin{itemize}
            \item<1-> Control de tensión en la red.
            \item<2-> Mejora calidad del suministro.
            \item<3-> Intercambio de potencia reactiva.
        \end{itemize}
        
        \vspace{0.5em}
        \onslide<4->{
        \textbf{Características:}
        \begin{itemize}
            \item Conexión en \textbf{\textcolor{accent}{paralelo}}.
            \item Basado en VSC (\textit{Voltage Source Converters}).
            \item Consume solo potencia activa para pérdidas.
        \end{itemize}
        }

        \column{0.45\textwidth}
        \begin{tikzpicture}[scale=0.85]
            % Red eléctrica (línea principal)
            \draw[line width=2pt] (-1,2) -- (4,2);
            \node[above] at (1.5,2) {\small Red};
            
            % STATCOM conectado en paralelo
            \onslide<4->{
            \draw[accent, line width=2pt] (2,2) -- (2,0);
            \draw[fill=secondary!30, draw=primary, line width=1.5pt] 
                (1,0) rectangle (3,-1.5);
            \node[font=\small\bfseries] at (2,-0.75) {STATCOM};
            
            % VSC interno
            \draw[fill=white] (1.3,-0.3) rectangle (2.7,-1.2);
            \node[font=\tiny] at (2,-0.75) {VSC};
            
            % Flechas de potencia reactiva
            \draw[-Stealth, success, line width=1.5pt] (2.5,1.8) -- (2.5,0.2);
            \draw[-Stealth, warning, line width=1.5pt] (1.5,0.2) -- (1.5,1.8);
            \node[font=\tiny, text width=1.5cm] at (3.5,1) {$Q$ reactiva};
            }
            
            % Tierra
            \draw (2,-1.5) -- (2,-2);
            \foreach \x in {-0.3,0,0.3}
                \draw (\x+2,-2) -- (\x+2,-2.2);
            \draw (1.4,-2.2) -- (2.6,-2.2);
        \end{tikzpicture}
        
        \vspace{0.5em}
        \onslide<4->{
        \begin{alertblock}{Clave}
            \centering Conexión en \textbf{derivación}
        \end{alertblock}
        }
    \end{columns}
\end{frame}

% --- SECCIÓN 3 ---
\section{Transformadores de Estado Sólido}

\begin{frame}{¿Qué son los SST?}
    \begin{block}{Definición}
        Los SST son equipos de paso que se conectan \textbf{\textcolor{accent}{en serie}} al flujo de potencia, similar a los transformadores convencionales, pero usando electrónica de potencia.
    \end{block}

    \vspace{1em}

    \begin{center}
    \begin{tikzpicture}[auto, node distance=2.5cm, >=Stealth, scale=1.1, transform shape]
        % Red MT
        \onslide<1->{
        \node[circle, draw=primary, line width=2pt, minimum size=1cm, fill=white] (redMT) {};
        \draw[primary, line width=1.5pt] (redMT.north) -- ++(0,0.3);
        \draw[primary, line width=1.5pt] (redMT.south) -- ++(0,-0.3);
        \node[below=0.3cm of redMT, font=\small] {Red MT};
        }
        
        % SST
        \onslide<2->{
        \node[right=1.5cm of redMT, fill=secondary!30, draw=primary, line width=2pt, 
              minimum width=3cm, minimum height=1.5cm, rounded corners] (sst) {};
        \node[font=\Large\bfseries, primary] at (sst) {SST};
        
        % Detalles internos del SST
        \draw[darkgray, line width=0.5pt] ([xshift=-0.8cm]sst.center) rectangle ++(0.4,0.6);
        \draw[darkgray, line width=0.5pt] ([xshift=-0.2cm]sst.center) rectangle ++(0.4,0.6);
        \draw[darkgray, line width=0.5pt] ([xshift=0.4cm]sst.center) rectangle ++(0.4,0.6);
        }
        
        % Red BT
        \onslide<1->{
        \node[circle, draw=primary, line width=2pt, minimum size=1cm, fill=white, right=1.5cm of sst] (redBT) {};
        \draw[primary, line width=1.5pt] (redBT.north) -- ++(0,0.3);
        \draw[primary, line width=1.5pt] (redBT.south) -- ++(0,-0.3);
        \node[below=0.3cm of redBT, font=\small] {Red BT};
        }
        
        % Conexiones en serie
        \onslide<3->{
        \draw[very thick, primary, -Stealth] (redMT) -- (sst) node[midway, above, font=\scriptsize] {Serie};
        \draw[very thick, primary, -Stealth] (sst) -- (redBT) node[midway, above, font=\scriptsize] {Paso};
        }
        
        % Comparación visual con STATCOM
        \onslide<4->{
        \node[below=1.5cm of sst, text width=6cm, align=center, font=\small] 
            {\textcolor{accent}{STATCOM} $\rightarrow$ Paralelo (Derivación) vs. 
             \textcolor{primary}{SST} $\rightarrow$ Serie (Paso)};
        }
    \end{tikzpicture}
    \end{center}
\end{frame}

\begin{frame}{Funcionalidad Híbrida}
    \begin{center}
        \begin{tikzpicture}
            \onslide<1->{
            % Ecuación visual
            \node[font=\Large, primary] (sst) at (0,0) {\textbf{SST}};
            \node[font=\Large] at (1.5,0) {=};
            }
            
            \onslide<2->{
            % Transformador
            \node[draw=secondary, fill=secondary!20, rounded corners, minimum width=2cm, 
                  minimum height=1cm, line width=1.5pt] (trafo) at (3.5,0) {};
            \node[font=\small, secondary] at (trafo) {\textbf{Transformador}};
            \draw[secondary, decoration={coil}, decorate, line width=1pt] 
                ([xshift=-0.3cm]trafo.west) -- ([xshift=-0.3cm, yshift=0.4cm]trafo.west);
            }
            
            \node[font=\Large] at (5.5,0) {+};
            
            \onslide<3->{
            % Compensador
            \node[draw=accent, fill=accent!20, rounded corners, minimum width=2cm, 
                  minimum height=1cm, line width=1.5pt] (comp) at (7.5,0) {};
            \node[font=\small, accent] at (comp) {\textbf{Compensador}};
            \draw[accent, -Stealth, line width=1pt] 
                ([xshift=0.2cm]comp.east) -- ([xshift=0.2cm, yshift=0.3cm]comp.east);
            \draw[accent, -Stealth, line width=1pt] 
                ([xshift=0.4cm, yshift=0.3cm]comp.east) -- ([xshift=0.4cm]comp.east);
            }
        \end{tikzpicture}
    \end{center}

    \vspace{2em}

    \begin{columns}[t]
        \column{0.45\textwidth}
        \onslide<2->{
        \begin{exampleblock}{Como Transformador}
            \begin{itemize}
                \item[\faCheckCircle] Transforma tensión.
                \item[\faCheckCircle] Transfiere potencia activa.
                \item[\faCheckCircle] Aislamiento galvánico.
            \end{itemize}
        \end{exampleblock}
        }

        \column{0.45\textwidth}
        \onslide<3->{
        \begin{alertblock}{Como Compensador}
            \begin{itemize}
                \item[\faCheckCircle] Control de factor de potencia.
                \item[\faCheckCircle] Regulación dinámica.
                \item[\faCheckCircle] Eliminación de armónicos.
            \end{itemize}
        \end{alertblock}
        }
    \end{columns}
\end{frame}

% --- SECCIÓN 4 ---
\section{Características Principales}

\begin{frame}{Características Esenciales de los SST}
    \begin{enumerate}
        \item<1-> \textbf{Transformación de tensión} con transmisión de potencia activa.
        
        \item<2-> \textbf{Control de factor de potencia} activo.
        
        \item<3-> \textbf{Regulación dinámica de tensión:}
        \begin{itemize}
            \item Si entrada baja 10\% $\rightarrow$ salida mantiene 230V exactos.
            \item \textcolor{darkgray}{(Un trafo convencional no puede hacer esto sin tap-changers lentos)}.
        \end{itemize}
        
        \item<4-> \textbf{Reducción drástica de peso y volumen:}
        \begin{itemize}
            \item Opera a $\sim$\textcolor{accent}{20,000 Hz} (vs 50 Hz convencional).
            \item Transformador físico mucho más pequeño.
        \end{itemize}
    \end{enumerate}
    
    \vspace{0.5em}
    \onslide<4->{
    \begin{center}
    \begin{tikzpicture}[scale=0.6]
        % Trafo convencional (grande)
        \draw[fill=darkgray!30, draw=darkgray, line width=1pt] 
            (0,0) rectangle (2,2.5);
        \node[font=\tiny, text width=1.5cm, align=center] at (1,-0.5) {Trafo 50Hz};
        \node[font=\scriptsize] at (1,1.25) {\Large \faIndustry};
        
        % vs
        \node[font=\small] at (3,1.25) {vs};
        
        % SST (pequeño)
        \draw[fill=secondary!30, draw=primary, line width=1pt] 
            (4,0.75) rectangle (5,1.75);
        \node[font=\tiny, text width=1.5cm, align=center] at (4.5,-0.5) {SST 20kHz};
        \node[font=\scriptsize] at (4.5,1.25) {\faMicrochip};
        
        % Flecha de comparación
        \draw[-Stealth, accent, line width=2pt] (2.2,2) -- (3.8,2);
        \node[accent, font=\small, above] at (3,2) {70\% más pequeño};
    \end{tikzpicture}
    \end{center}
    }
\end{frame}

\begin{frame}{Más Características Destacadas}
    \begin{columns}
        \column{0.6\textwidth}
        \begin{enumerate}
            \setcounter{enumi}{4}
            \item<1-> \textbf{Bus de corriente continua (DC):}
            \begin{itemize}
                \footnotesize
                \item Conexión directa de paneles solares.
                \item Carga directa de VE sin rectificador externo.
            \end{itemize}
            \item<2-> \textbf{Eliminación de armónicos.}
            \item<3-> \textbf{Clave para SmartGrids:}
            \begin{itemize}
                 \footnotesize
                \item Control total del flujo (bidireccional).
            \end{itemize}
        \end{enumerate}
        
        \column{0.4\textwidth}
        \onslide<1->{
        \begin{tikzpicture}[scale=0.7]
            % Bus DC central
            \draw[accent, line width=3pt] (0,2) -- (4,2);
            \node[accent, font=\small] at (2,2.5) {Bus DC};
            
            % Panel solar
            \draw[fill=warning!30, draw=warning, line width=1pt] 
                (0.5,3) rectangle (1.5,3.8);
            \node[font=\tiny] at (1,3.4) {\faSolarPanel};
            \draw[-Stealth, success, line width=1.5pt] (1,3) -- (1,2);
            
            % Batería
            \draw[fill=success!30, draw=success, line width=1pt] 
                (2.5,3) rectangle (3.5,3.8);
            \node[font=\tiny] at (3,3.4) {\faBatteryFull};
            \draw[-Stealth, success, line width=1.5pt] (3,2) -- (3,3);
            
            % Coche eléctrico
            \draw[fill=primary!30, draw=primary, line width=1pt] 
                (1,0.5) rectangle (3,1.2);
            \node[font=\tiny] at (2,0.85) {\faCar};
            \draw[-Stealth, primary, line width=1.5pt] (2,2) -- (2,1.2);
        \end{tikzpicture}
        }
        
        \vspace{1em}
        \onslide<3->{
        \begin{alertblock}{Aplicación}
            Principalmente en redes de \textbf{Media} y \textbf{Baja Tensión} (límites actuales de semiconductores).
        \end{alertblock}
        }
    \end{columns}
\end{frame}

% --- SECCIÓN 5 ---
\section{Arquitectura del SST}

\begin{frame}{Esquema Típico de un SST}
    \begin{center}
    \begin{tikzpicture}[
        scale=0.9,
        block/.style={rectangle, draw=primary, line width=1.5pt, fill=white, 
                      minimum width=2cm, minimum height=1.2cm, rounded corners, font=\small},
        label/.style={font=\tiny, text=darkgray}
    ]
        % Etapa 1: Rectificador
        \onslide<1->{
        \node[block, fill=secondary!20] (rect) at (0,0) {AC/DC};
        \node[label, above=0.1cm of rect] {Etapa 1};
        \node[label, below=0.1cm of rect, text width=2cm, align=center] {Rectificador\\Control FP};
        }
        
        % Etapa 2: DC/DC con HFT
        \onslide<2->{
        \node[block, fill=accent!20, right=1.5cm of rect] (dcdc) at (3.5,0) {DC/DC};
        \node[label, above=0.1cm of dcdc] {Etapa 2};
        \node[label, below=0.1cm of dcdc, text width=2cm, align=center] {HFT\\$\sim$20 kHz};
        
        % Destacar el HFT
        \draw[accent, line width=2pt, dashed, rounded corners] 
            ([xshift=-0.3cm, yshift=0.8cm]dcdc.north west) rectangle 
            ([xshift=0.3cm, yshift=-0.8cm]dcdc.south east);
        \node[accent, font=\footnotesize, above=1cm of dcdc] {\faHeart\ Corazón del SST};
        }
        
        % Etapa 3: Inversor
        \onslide<3->{
        \node[block, fill=secondary!20, right=1.5cm of dcdc] (inv) at (7,0) {DC/AC};
        \node[label, above=0.1cm of inv] {Etapa 3};
        \node[label, below=0.1cm of inv, text width=2cm, align=center] {Inversor\\50 Hz};
        }
        
        % Conexiones
        \onslide<1->{
        \draw[-Stealth, thick, primary] (-2,0) -- (rect) node[midway, above, font=\scriptsize] {MT AC};
        }
        \onslide<2->{
        \draw[-Stealth, thick, primary] (rect) -- (dcdc) node[midway, above, font=\tiny] {DC};
        }
        \onslide<3->{
        \draw[-Stealth, thick, primary] (dcdc) -- (inv) node[midway, above, font=\tiny] {DC};
        \draw[-Stealth, thick, primary] (inv) -- ++(2,0) node[midway, above, font=\scriptsize] {BT AC};
        }
        
        % Bus DC
        \onslide<4->{
        \draw[accent, line width=2pt] (0,-1.5) -- (7,-1.5);
        \node[accent, font=\small, below] at (3.5,-1.5) {Bus DC común};
        \draw[-Stealth, accent, thick] (3.5,-1.5) -- (3.5,-0.7);
        }
    \end{tikzpicture}
    \end{center}
    
    \vspace{1em}
    \onslide<4->{
    \begin{alertblock}{Ventaja Clave}
        Las tres etapas conectadas en serie permiten \textbf{control total} y un \textbf{transformador físico miniaturizado}.
    \end{alertblock}
    }
\end{frame}

\begin{frame}{Etapa 2: El "Corazón" (DC/DC Alta Frecuencia)}
    \begin{columns}
        \column{0.55\textwidth}
        \onslide<1->{
        \begin{block}{Componentes}
        \begin{itemize}
            \item Inversor DC/AC (alta frecuencia).
            \item \textbf{Transformador de Alta Frecuencia (HFT)}.
            \item Rectificador AC/DC.
        \end{itemize}
        \end{block}
        }
        
        \vspace{0.5em}
        \onslide<2->{
        \textbf{Ventaja del HFT:}
        \small
        Al operar a $\sim$20 kHz, el núcleo magnético necesario es \textbf{minúsculo} comparado con uno de 50 Hz, manteniendo el aislamiento galvánico.
        }
        
        \vspace{0.5em}
        \onslide<3->{
        \begin{exampleblock}{La Física detrás}
            $$B = \frac{V}{4.44 \cdot f \cdot N \cdot A}$$
            \footnotesize
            Mayor frecuencia $\rightarrow$ menor área del núcleo.
        \end{exampleblock}
        }
        
        \column{0.45\textwidth}
        \begin{tikzpicture}[scale=0.8]
            % Comparación visual de tamaños
            \onslide<2->{
            % Transformador 50 Hz (grande)
            \draw[fill=darkgray!40, draw=darkgray, line width=2pt] 
                (0,0) rectangle (2,3);
            \node[font=\small, text width=1.5cm, align=center] at (1,1.5) {50 Hz};
            \node[font=\footnotesize] at (1,-0.6) {Trafo convencional};
            
            % Flecha
            \draw[-Stealth, accent, line width=3pt] (2.5,1.5) -- (3.5,1.5);
            \node[accent, font=\tiny, above] at (3,1.5) {400x más frecuencia};
            }
            
            \onslide<3->{
            % Transformador 20 kHz (pequeño)
            \draw[fill=success!40, draw=success, line width=2pt] 
                (4,0.8) rectangle (5,2.2);
            \node[font=\small, text width=1.2cm, align=center] at (4.5,1.5) {20 kHz};
            \node[font=\footnotesize, success] at (4.5,-0.6) {HFT del SST};
            
            % Destacar reducción
            \node[font=\Large, success] at (4.5,3.2) {\faCheckCircle};
            \node[font=\tiny, text width=1.8cm, align=center, success] at (4.5,2.7) {95\% más pequeño};
            }
        \end{tikzpicture}
    \end{columns}
\end{frame}

\begin{frame}{Proceso de Conversión Completo}
    \centering
    
    % Diagrama de flujo mejorado con animación
    \begin{tikzpicture}[
        node distance=0.3cm and 0.5cm,
        auto,
        block/.style={rectangle, draw=primary, thick, fill=white, text width=1.6cm, 
                      align=center, rounded corners, minimum height=1.2cm, font=\scriptsize,
                      drop shadow},
        arrow/.style={-Stealth, line width=2pt, primary},
        stage/.style={font=\tiny, darkgray}
    ]
        % Entrada
        \node[text width=1.2cm, align=center] (in) {\textbf{MT}\\50Hz AC};
        
        % Etapa 1
        \onslide<1->{
        \node[block, fill=secondary!20, right=0.5cm of in] (rect) {Rectific.\\AC/DC};
        \node[stage, below=0.05cm of rect] {Etapa 1};
        \draw[arrow] (in) -- (rect);
        }
        
        % A DC
        \onslide<2->{
        \node[right=0.3cm of rect, font=\tiny, text=accent] (dc1) {DC};
        \draw[arrow] (rect) -- (dc1);
        }
        
        % Etapa 2a: Inversor HF
        \onslide<3->{
        \node[block, fill=warning!20, right=0.3cm of dc1] (invHF) {Inv. HF\\DC/AC};
        \node[stage, below=0.05cm of invHF] {2a};
        \draw[arrow] (dc1) -- (invHF);
        }
        
        % A AC HF
        \onslide<4->{
        \node[right=0.3cm of invHF, font=\tiny, text=accent, text width=0.8cm, align=center] (achf1) {AC\\20kHz};
        \draw[arrow] (invHF) -- (achf1);
        }
        
        % Etapa 2b: HFT
        \onslide<5->{
        \node[block, fill=accent!30, draw=accent, right=0.3cm of achf1] (trafo) {\textbf{HFT}\\Transf.};
        \node[stage, below=0.05cm of trafo, text=accent] {2b \faHeart};
        \draw[arrow, accent] (achf1) -- (trafo);
        }
        
        % AC HF BT
        \onslide<6->{
        \node[right=0.3cm of trafo, font=\tiny, text=accent, text width=0.8cm, align=center] (achf2) {AC BT\\20kHz};
        \draw[arrow, accent] (trafo) -- (achf2);
        }
        
        % Etapa 2c: Rectificador HF
        \onslide<7->{
        \node[block, fill=warning!20, right=0.3cm of achf2] (rectHF) {Rect. HF\\AC/DC};
        \node[stage, below=0.05cm of rectHF] {2c};
        \draw[arrow] (achf2) -- (rectHF);
        }
        
        % A DC
        \onslide<8->{
        \node[right=0.3cm of rectHF, font=\tiny, text=accent] (dc2) {DC};
        \draw[arrow] (rectHF) -- (dc2);
        }
        
        % Etapa 3: Inversor final
        \onslide<9->{
        \node[block, fill=secondary!20, right=0.3cm of dc2] (inv) {Inversor\\DC/AC};
        \node[stage, below=0.05cm of inv] {Etapa 3};
        \draw[arrow] (dc2) -- (inv);
        }
        
        % Salida
        \onslide<10->{
        \node[text width=1.2cm, align=center, right=0.5cm of inv] (out) {\textbf{BT}\\50Hz AC};
        \draw[arrow] (inv) -- (out);
        }
    \end{tikzpicture}

    \vspace{1.5em}
    \onslide<11->{
    \begin{alertblock}{La Clave}
        Pasar por \textbf{Alta Frecuencia} intermedia permite reducir el tamaño físico drásticamente manteniendo aislamiento galvánico.
    \end{alertblock}
    }
\end{frame}

% --- SECCIÓN 6 ---
\section{Ventajas y Desafíos}

\begin{frame}{Resumen: Pros y Contras}
    \begin{columns}[t]
        \column{0.48\textwidth}
        \textbf{\textcolor{success}{Ventajas}} \Large\textcolor{success}{\faThumbsUp}
        \normalsize
        \vspace{0.5em}
        \begin{itemize}
            \item<1->[\textcolor{success}{\faCheckCircle}] Tamaño y peso reducidos (70-95\%).
            \item<2->[\textcolor{success}{\faCheckCircle}] Control instantáneo de voltaje.
            \item<3->[\textcolor{success}{\faCheckCircle}] Bus DC integrado (Renovables/VE).
            \item<4->[\textcolor{success}{\faCheckCircle}] Bidireccionalidad total.
            \item<5->[\textcolor{success}{\faCheckCircle}] Eliminación de armónicos.
            \item<6->[\textcolor{success}{\faCheckCircle}] Ideal para SmartGrids.
        \end{itemize}

        \column{0.48\textwidth}
        \textbf{\textcolor{accent}{Desafíos}} \Large\textcolor{accent}{\faExclamationTriangle}
        \normalsize
        \vspace{0.5em}
        \begin{itemize}
            \item<7->[\textcolor{accent}{\faTimesCircle}] \textbf{Coste elevado:} Semiconductores (SiC, GaN).
            \item<8->[\textcolor{accent}{\faTimesCircle}] \textbf{Pérdidas:} Ligeramente mayores (etapas múltiples).
            \item<9->[\textcolor{accent}{\faTimesCircle}] \textbf{Limitación de tensión:} Solo MT/BT actualmente.
            \item<10->[\textcolor{accent}{\faTimesCircle}] \textbf{Pérdida de inercia:} Sin masa rotatoria, menor estabilidad ante perturbaciones.
        \end{itemize}
    \end{columns}
    
    \vspace{1em}
    \onslide<11->{
    \begin{center}
    \begin{tikzpicture}
        % Escala visual de madurez tecnológica
        \draw[line width=1pt, darkgray] (0,0) -- (10,0);
        \foreach \x in {0,2.5,5,7.5,10}
            \draw[darkgray] (\x,0) -- (\x,-0.2);
        
        \node[below, font=\tiny] at (0,-0.2) {Laboratorio};
        \node[below, font=\tiny] at (2.5,-0.2) {Prototipos};
        \node[below, font=\tiny] at (5,-0.2) {Primeras aplicaciones};
        \node[below, font=\tiny] at (7.5,-0.2) {Comercial};
        \node[below, font=\tiny] at (10,-0.2) {Estándar};
        
        % Posición actual del SST
        \fill[accent] (5,0) circle (0.2cm);
        \node[accent, above, font=\small] at (5,0.3) {SST hoy};
        \draw[-Stealth, accent, line width=1.5pt] (5.5,0.3) -- (7,0.3);
    \end{tikzpicture}
    \end{center}
    }
\end{frame}

\begin{frame}{Comparativa Visual: Trafo vs SST}
    \begin{center}
    \begin{tikzpicture}[scale=0.9]
        % Tabla comparativa
        \node[font=\small\bfseries] at (2,4) {Transformador Convencional};
        \node[font=\small\bfseries] at (7,4) {SST};
        
        % Filas de comparación
        \node[font=\small, left] at (0,3.2) {Peso:};
        \node[font=\small] at (2,3.2) {100\%};
        \node[font=\small] at (7,3.2) {5-30\%};

        \node[font=\small, left] at (0,2.4) {Volumen:};
        \node[font=\small] at (2,2.4) {100\%};
        \node[font=\small] at (7,2.4) {10-30\%};

        \node[font=\small, left] at (0,1.6) {Regulación:};
        \node[font=\small] at (2,1.6) {Tap manual};
        \node[font=\small] at (7,1.6) {Instantánea};

        \node[font=\small, left] at (0,0.8) {Eficiencia:};
        \node[font=\small] at (2,0.8) {99.5\%};
        \node[font=\small] at (7,0.8) {>97\%};

        \node[font=\small, left] at (0,0) {Coste:};
        \node[font=\small] at (2,0) {Bajo};
        \node[font=\small] at (7,0) {Alto};

        % Indicadores visuales
        \onslide<2->{
        \node[success] at (9,3.2) {\faArrowUp};
        \node[success] at (9,2.4) {\faArrowUp};
        \node[success] at (9,1.6) {\faArrowUp};
        \node[warning] at (9,0.8) {\faArrowDown};
        \node[accent] at (9,0) {\faArrowDown};
        }
        
        % Líneas separadoras
        \draw[darkgray, dashed] (0,3.6) -- (9,3.6);
        \draw[darkgray, dashed] (0,2.8) -- (9,2.8);
        \draw[darkgray, dashed] (0,2) -- (9,2);
        \draw[darkgray, dashed] (0,1.2) -- (9,1.2);
        \draw[darkgray, dashed] (0,0.4) -- (9,0.4);
    \end{tikzpicture}
    \end{center}
    
    \vspace{0.5em}
    \onslide<3->{
    \begin{alertblock}{Conclusión Comparativa}
        El SST supera al transformador convencional en \textbf{compacidad, control y funcionalidad}, pero requiere reducción de costes para despliegue masivo.
    \end{alertblock}
    }
\end{frame}

% --- CIERRE ---
\section{Conclusiones}

\begin{frame}{Conclusiones}
    \begin{center}
    \begin{tikzpicture}[scale=0.9]
        % Diagrama de síntesis
        \onslide<1->{
        \node[draw=primary, fill=primary!10, line width=2pt, rounded corners, 
              minimum width=8cm, minimum height=1.5cm, font=\large] (pot) at (0,2) 
            {\textbf{Potencial Transformador de los SST}};
        }
        
        % Características clave
        \onslide<2->{
        \node[draw=secondary, fill=white, rounded corners, text width=2.5cm, align=center] (c1) at (-3,0) 
            {\small Transformación + Compensación};
        \draw[-Stealth, thick, primary] (pot) -- (c1);
        }
        
        \onslide<3->{
        \node[draw=secondary, fill=white, rounded corners, text width=2.5cm, align=center] (c2) at (0,0) 
            {\small Compacto e Inteligente};
        \draw[-Stealth, thick, primary] (pot) -- (c2);
        }
        
        \onslide<4->{
        \node[draw=secondary, fill=white, rounded corners, text width=2.5cm, align=center] (c3) at (3,0) 
            {\small SmartGrids del futuro};
        \draw[-Stealth, thick, primary] (pot) -- (c3);
        }
    \end{tikzpicture}
    \end{center}

    \vspace{1em}
    \onslide<5->{
    \textbf{Hoja de ruta:}
    \begin{itemize}
        \item[\faFlask] Tecnología probada pero cara.
        \item[\faMicrochip] Necesario reducir costes de carburo de silicio (SiC) y GaN.
        \item[\faCity] Futuro estándar para las \textit{Smart Cities}.
        \item[\faChartLine] Aplicaciones nicho: VE, renovables, microredes.
    \end{itemize}
    }
    
    \vspace{0.5em}
    \onslide<6->{
    \begin{exampleblock}{Visión 2030-2040}
        Los SST serán el \textbf{estándar} en distribución urbana, integrando generación distribuida, almacenamiento y movilidad eléctrica en un solo dispositivo inteligente.
    \end{exampleblock}
    }
\end{frame}

\begin{frame}{Aplicaciones Actuales y Futuras}
    \begin{columns}
        \column{0.5\textwidth}
        \textbf{\textcolor{primary}{Aplicaciones actuales:}}
        \begin{itemize}
            \item<1->[\faBolt] Estaciones de carga ultrarrápida VE.
            \item<2->[\faSolarPanel] Microredes con renovables.
            \item<3->[\faIndustry] Industrias con calidad crítica.
            \item<4->[\faShip] Tracción naval/ferroviaria.
        \end{itemize}
        
        \column{0.5\textwidth}
        \textbf{\textcolor{accent}{Futuro cercano:}}
        \begin{itemize}
            \item<5->[\faCity] Distribución urbana inteligente.
            \item<6->[\faBuilding] Edificios con gestión energética.
            \item<7->[\faServer] Centros de datos eficientes.
            \item<8->[\faPlane] Aviación eléctrica (potencia).
        \end{itemize}
    \end{columns}
    
    \vspace{1.5em}
    \onslide<9->{
    \begin{center}
    \begin{tikzpicture}
        % Timeline visual
        \draw[-Stealth, thick, primary] (0,0) -- (8,0);
        \node[below, font=\tiny] at (0,0) {2024};
        \node[below, font=\tiny] at (4,0) {2030};
        \node[below, font=\tiny] at (8,0) {2040};
        
        % Hitos
        \fill[accent] (2,0) circle (0.15cm);
        \node[above, font=\tiny, text width=2cm, align=center] at (2,0.3) {Primeras instalaciones comerciales};
        
        \fill[success] (5,0) circle (0.15cm);
        \node[above, font=\tiny, text width=2cm, align=center] at (5,0.3) {Reducción coste 50\%};
        
        \fill[primary] (7.5,0) circle (0.15cm);
        \node[above, font=\tiny, text width=2cm, align=center] at (7.5,0.3) {Estándar en ciudades};
    \end{tikzpicture}
    \end{center}
    }
\end{frame}

\begin{frame}[standout]
    \Huge ¿Preguntas?
    
    \vspace{2em}
    
    \begin{tikzpicture}
        % Icono de email
        \node[font=\Large, primary] at (0,0) {\faEnvelope};
        \node[font=\normalsize, right] at (0.5,0) {email@instituto.com};
    \end{tikzpicture}
    
    \vspace{1em}
    \normalsize
    \textcolor{darkgray}{Gracias por vuestra atención}
\end{frame}

\end{document}